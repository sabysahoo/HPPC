\documentclass[conference]{IEEEtran}
\usepackage{cite}
\usepackage{tcucosc}
\usepackage{lipsum}


\title{Parallelized Traffic Simulation in C++}

\author{
\IEEEauthorblockN{Zach Alaniz}
\IEEEauthorblockA{Department of Computer Science\\
College of Science and Engineering\\
Texas Christian University\\
Fort Worth, Texas 76129\\
Email: z.a.alaniz@tcu.edu\\}
\and
\IEEEauthorblockN{Saby Sahoo}
\IEEEauthorblockA{Department of Computer Science\\
College of Science and Engineering\\
Texas Christian University\\
Fort Worth, Texas 76129\\
Email: s.sahoo@tcu.edu\\}
\and
\IEEEauthorblockN{Bradley Schoeneweis}
\IEEEauthorblockA{Department of Computer Science\\
College of Science and Engineering\\
Texas Christian University\\
Fort Worth, Texas 76129\\
Email: b.schoeneweis@tcu.edu\\}
}



\begin{document}


\maketitle


\begin{abstract}
This semester research project investigates the potential speedups and scalability of running a traffic simulation in parallel using C++ and OpenMP. These speedups are evaluated by comparing the sequential performance to the parallel performance when varying thread counts for a fixed-size, randomized input. While a large percentage of the program is inherently sequential due to computation, the process of transitioning from one state to another was parallelized.
\end{abstract}
\bigskip
\begin{IEEEkeywords}
OpenMP, Cellular Automata, Parallel Simulation, C++
\end{IEEEkeywords}

\section{Introduction}
Extensive research and improvement has been made over the years in the field of study that is parallel computing. Falling within this domain of study is the topic of simulation, and more specifically, traffic simulation. The ability to optimize light scheduling and move cars from one point to another in an efficient, organized manner is paramount to day-to-day travel. Being able to apply parallel computing to simulate increasingly large traffic environments leads to many interesting research topics and advances in road safety, traffic capacity, economic impact, among many others. The topic of simulating traffic flow for increasingly large sizes of roads is the topic of interest in the following research. Real life traffic flow allows for movement when possible at all times, and by utilizing parallel computing topics and tools, the speedup of a traffic simulation should prove beneficial in seeing how a particular road and intersection works for an increasingly large amount of cars funneling in and out. 


\section{Background and Related Research}
\lipsum[2] \cite{Laszlo1996}

\section{Parallelization Technique}
\lipsum[3] \cite{ORourke2005}

\section{Summary of Results}
\lipsum[4]

\section{Conclusions, Lessons Learned, and Future Work}
\lipsum[5]

\bibliographystyle{ieeetr}
\bibliography{HPC_Project}

\end{document}